\documentclass[a4paper]{article}

\usepackage[english]{babel}
\usepackage[utf8]{inputenc}
\usepackage{amsmath}
\usepackage{graphicx}
\usepackage[colorinlistoftodos]{todonotes}


\title{Biogears in Anesthesia Screen Based Simulation}

\author{
  Xiao Han\\
  \and Liewei Ke
  \and Guan Wang
}

\date{\today}

\begin{document}
\maketitle

\begin{abstract}
Abstract.
\end{abstract}

\section{Current physiologic engines and BioGears}

\section{Show equivalence to current training methods}

- how to show current training methods
- do we need to include any human body experiments for comparison

\subsection{Current training methods}
\subsection{Why using screen based simulation is equivalent to current training methods}

\section{Show physiologically appropriate, responses to key events}

- bleeding out scenarios, multiple experiments, (mean, root mean squared error, median performance error data recorded)
- oxygen desaturation (same as aboved)

- patient stop breathing for 30 seconds, 2 mins and recover

- (optional) compare with another dataset asac dataset

- simulate with propofal(200mg), morphine so4(20mg), rocuromium(50mg), succinocloon(100mg), room air(170/70kg)

\subsection{Physical experiments}
\subsection{Other simulators}

\section{Enumerate limits and future work}
\subsection{Limits of screen based simulation}
\subsection{Future work}

\end{document}